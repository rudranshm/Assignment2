\documentclass[journal,12pt,twocolumn]{IEEEtran}
\usepackage{enumitem}
\usepackage{amsmath}
\usepackage{amssymb}
\usepackage{graphicx}


\title{Assignment 2 \\ \Large AI1110: Probability and Random Variables \\ \large Indian Institute of Technology Hyderabad}
\author{Rudransh Mishra \\ \normalsize AI21BTECH11025 \\ \vspace*{20pt} \normalsize  28 April 2022 \\ \vspace*{20pt} \Large ICSE 2017 Grade 12}


\begin{document}
% The title
\maketitle

% The question
\textbf{Question 1(ii)} 
A problem is given to three students whose chances of solving it are \(\frac{1}{4}\), \(\frac{1}{5}\) and \(\frac{1}{3}\) respectively. Find the probability that the problem is solved.

% The solution

\textbf{Solution.}\\
\noindent Probability of $1^{st}$ student solving the problem, P1: \(\frac{1}{4}\)\\
\\
Probability of $2^{nd}$ student solving the problem, P2: \(\frac{1}{5}\)\\
\\
Probability of $3^{rd}$ student solving the problem, P3: \(\frac{1}{3}\)\\

The probability of event E not occuring, $\neg E$ , is given as :
\begin{align}
  &\neg E = 1 - E
\end{align}
$\therefore$ The probabilty of the problem being solved, P(Solved) will be given by
\begin {align}
  &P(Solved) = 1-P(NotSolved) ,
\end{align}
Where P(NotSolved) is the probability of no student solving the problem.\\

\noindent Let the probability of individual students not solving the problem be $\neg P1$, $\neg P2$, $\neg P3$ respectively. Question is unsolved if all three students simultaneously do not solve it.\\

\noindent $\therefore$ P(NotSolved) can be give as:
\begin{align}
   &P(NotSolved) = \neg P1 \times \neg P2 \times \neg P3 
\end{align}
\noindent $\therefore$ By equations (1) and (3),
\begin{align}
  &P(NotSolved) = (1 - P1) \times (1 - P2) \times (1 - P3)
\end{align}
\noindent Substituting with values for P1,P2,P3,
\begin{align}
  $P(NotSolved) = $(1 - \(\frac{1}{4}\)) \times (1 - \(\frac{1}{5}\)) \times (1 - \(\frac{1}{3}\))$
\end{align}
\begin{align}
$= \(\frac{3}{4}\)\times\(\frac{4}{5}\)\times\(\frac{2}{3}\) $ 
$= \(\frac{2}{5}\) $
$= 0.4$
\end{align}

\noindent $\therefore$ By equation (2), 
\begin {align}
  &P(Solved) = 1-P(NotSolved)\\
  &P(Solved) = 1-0.4\\
  &P(Solved) = 0.6
\end{align}

Thus, the probabilty of the problem being solved, P(Solved) will be 0.6.\\

\begin{tabular}{ ||p{3.5cm}|p{2cm}||  }
 \hline
 \hline
 \multicolumn{2}{||c||}{\textbf{Input and Output}} \\
 \hline
 \hline
 \textbf{Input}& \textbf{Value} \\
 \hline
 &\\
 P1 &   \(\frac{1}{4}\)\\
 &\\
 \hline
 &\\
 P2 &   \(\frac{1}{5}\)\\
 &\\
 \hline
 &\\
 P3 &   \(\frac{1}{3}\)\\
 &\\
 \hline
 \textbf{Output}& \textbf{Value} \\
 \hline
 P(solved)&   0.6 \\
 \hline
 \hline
\end{tabular}
\end{document}